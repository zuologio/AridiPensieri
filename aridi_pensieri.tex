% !TeX program = xelatex

\documentclass[10pt]{book}

% ---------- PACCHETTI ----------
\usepackage[paperwidth=4.25in, paperheight=7in,
            inner=0.8in, outer=0.6in,
            top=0.7in, bottom=0.7in]{geometry}

\usepackage{fontspec}      % Per XeLaTeX
\usepackage{microtype}     % Migliore tipografia
\usepackage{setspace}      % Per gestire interlinea
\usepackage{titling}       % Per personalizzare frontespizio
\usepackage{verse}         % Per poesie
\usepackage{lipsum}        % (solo debug, puoi rimuoverlo)
\usepackage{fancyhdr}      % Per footer e header
\usepackage{emptypage}     % Evita numerazione su pagine vuote
\usepackage[italian]{babel}

% ---------- FONT ----------
%\setmainfont{Cormorant Garamond}
%\setmainfont{EB Garamond}
%\setmainfont{Crimson Pro}

\linespread{1.15}

% ---------- HEADER/FOOTER ----------
\pagestyle{plain}

% ---------- FRONTESPIZIO ----------
\title{\Huge Aridi Pensieri}
\author{\Large \emph{Raccolta di poesie}}
\date{}

% ---------- INIZIO DOCUMENTO ----------
\begin{document}

% ---------- PAGINA DEL TITOLO ----------
\maketitle
\thispagestyle{empty}
\clearpage

% ---------- PAGINA DELLA DEDICA ----------
\cleardoublepage
\thispagestyle{empty}

\vspace*{0.55\textheight}

\begin{flushright}
\itshape
A Sara,\\
per questa vita insieme
\end{flushright}

\clearpage

% ---------- NOTA DELL'AUTORE ----------
\cleardoublepage
\chapter*{Nota dell'autore}
\addcontentsline{toc}{chapter}{Nota dell'autore}

Aridi Pensieri raccoglie i testi di un decennio liminale, in bilico tra adolescenza e prima maturità, 
quando il mondo interiore si presenta nella sua forma più nuda e incandescente. 
Sono poesie nate in anni in cui l’identità si scompone e si ricompone continuamente, sospesa tra la ricerca 
di un’origine e l’intuizione, ancora incerta, di un approdo possibile. L’amore vi appare nella sua stagione più assoluta, 
privo di misura; il dolore, nella sua immediatezza senza riparo; le domande, nella loro necessità irrevocabile.
\vspace{1em}

Il percorso che attraversa la raccolta è quello di un io poetico che scopre la propria voce a partire dall’inquietudine, 
dalle soglie infrante, dai passaggi obbligati dell’esistenza. Ogni testo è una fenditura attraverso cui filtrano fragilità, 
desiderio di significato, spaesamento e un bisogno ostinato di chiarore. Non c’è compiacimento, ma l’urgenza autentica 
di dare forma a ciò che la vita, a quell’età, impone di sentire fino all’estremo.
Tra i luoghi simbolici di questo cammino emerge, con forza costante, la natura. 
Non come scenario, ma come struttura profonda dell’immaginario. 
La campagna veneta, con la sua ciclicità millenaria, diventa linguaggio e controcanto: 
i campi che riposano e si riaccendono, le nebbie che avvolgono e cancellano, la terra che custodisce e trattiene, 
le stagioni che avanzano con una sapienza che non conosce esitazioni. È una geografia 
dell’anima prima ancora che un paesaggio reale, un orizzonte dove il tempo umano si specchia nel tempo della terra, 
riconoscendosi fragile e al tempo stesso necessario.
\vspace{1em}

Aridi Pensieri non cerca soluzioni, né vuole offrire risposte. 
Preferisce restare nell’interstizio tra ciò che ferisce e ciò che salva, conservando la voce ruvida e 
assoluta degli anni in cui ogni emozione è un varco e ogni parola un tentativo di attraversarlo. 
È un libro che porta con sé la promessa, mai del tutto compiuta e proprio per questo vitale, 
di una verità ancora in movimento.

\clearpage

% ---------- INIZIO RACCOLTA ----------
\cleardoublepage
\chapter*{Poesie}
\addcontentsline{toc}{chapter}{Poesie}

% ----- POESIE -----
%
\cleardoublepage
\section*{Aridi Pensieri}
\vspace{2em}
\begin{verse}
Come aratro\\
imprimo solchi\\
nella mente\\
rivoltando le zolle\\
di aridi pensieri\\
imprigionato\\
in un giogo
malinconia...
\end{verse}

\cleardoublepage
\section*{La tua mano}
\vspace{2em}
\begin{verse}
Interprete smarrito\\
nei tuoi occhi tempestosi\\
inseguo errante\\
la linea immota\\
di un sorriso sepolto,\\
indecifrabile glifo\\
in cui cerco schegge\\
d'un illuminante amore\\
per trovarti oltre\\
questo nero mare di paure.\\
\vspace{1em}
Vorrei seguire le orme\\
del tuo solitario incedere\\
in spiagge remote\\
dove sola sei vagabonda,\\
camminando radente alle acque\\
che con amare carezze\\
d'ogni ondata di passione\\
coprono ogni traccia di te.\\
\vspace{1em}
E cade la notte\\
tra le grida silenziose\\
che soffochiamo\\
nel nostro mare orgoglio\\
ci siamo perduti\\
nella spiaggia sorda\\
sfiniti d’un amore\\
fatto di silenzi.\\
\vspace{1em}
Lontano, solo,\\
mi abbandono alla sabbia\\
e aprendo le braccia\\
per accogliere la notte\\
trovo nell’oscurità\\
la tua mano.\\
Tremante come foglia\\
rapita dal vento d'autunno\\
ti copro d'amore e di me.\\
Protezione, dolcezza\\
e respiro caldo su di te.
\end{verse}

\cleardoublepage
\section*{Cremisi Bagliore}
\vspace{2em}
\begin{verse}
Plasmiamo sogni\\
oltre l'umana misura\\
nel massificante\\
vivere inconscio\\
oltrepassando ogni limite\\
chiusi nell'unico limite\\
E tramontiamo inesorabilmente\\
oltre l'orizzonte\\
delle nostre certezze\\
disarmante fragilità\\
in un cremisi bagliore\\
nell'universo della storia\\
che si mescola\\
al viola serotino\\
della notte che divora
\end{verse}

\cleardoublepage
\section*{Falena}
\vspace{2em}
\begin{verse}
Con trasognato gradiente\\
verso il reciproco annientarci.\\
\vspace{1em}
A te per anni acclimatato,\\
come crisalide, appeso\\
al fulcro del mio universo\\
impercettibilmente in vita,\\
rifugiato\\
nel mio ego avviluppato\\
in un'insignificante bozzolo\\
che nemmeno io\\
avrei, privato d'ogni forza,\\
potuto estirpare...\\
\vspace{1em}
ameno pensiero d'estate,\\
come velluto di labbra\\
che vorrei sfiorare,\\
disperdendo timori di ieri,\\
dimostri\\
per induzione matematica\\
sul numero dei miei errori,\\
che l'invariante\\
del mio ciclo vitale\\
è comunque l'amore...\\
\vspace{1em}
Rinato con vestigia di falena\\
in una calda sera di giugno.\\
\end{verse}

\cleardoublepage
\section*{Grano al vento}
\vspace{2em}
\begin{verse}
Nella valle dipinta\\
dal cremisi crepuscolo\\
osservo il vento accarezzare\\
gli steli all'unisono\\
del grano maturo\\
\vspace{1em}
Come ogni spiga\\
nel dorato oceano\\
siamo cullati dolcemente\\
e piegati brutalmente\\
dal vento dell'amore
\end{verse}

\cleardoublepage
\section*{Giorni come punti}
\vspace{2em}
\begin{verse}
Linee alla ricerca\\
di direzioni cartesiane\\
dove proiettare il senso\\
del nostro respirare,\\
anti progettuali essenze\\
nella danzante varianza\\
del quotidiano cambiare\\
parallele come binari\\
di treni che mai prenderemo\\
verso luoghi vissuti\\
solo in sogni confusi,\\
oggi ci intersechiamo\\
nella finita sequenza\\
di giorni come punti\\
sul piano dell'esistenza,\\
due punti sovrapposti\\
un unico geometrico ego.
\end{verse}

\cleardoublepage
\section*{Matrice d'illusioni}
\vspace{2em}
\begin{verse}
Cadenzati dalla futilità\\
nella landa desolata\\
ammassati marciando,\\
volti spenti anelanti\\
alla proiezione conformista\\
di logica annichilente\\
pianificata in menti perverse.\\
\vspace{1em}
Oscure simmetrie\\
in miraggi di modernità\\
matrice d'illusioni\\
e ossimorica libertà.\\
\vspace{1em}
Egocentrica preghiera\\
in un ecosistema schermato\\
da una superficie speculare\\
nella quale ricorsivamente\\
a se stessi tornare\\
nel bisogno di soddisfare\\
ogni nuovo innaturale bisogno.\\
\end{verse}

\cleardoublepage
\section*{Ritornando a casa}
\vspace{2em}
\begin{verse}
Luci artificiali\\
penetrano dai vetri\\
rompendo il nero,\\
nella corriera rumorosa\\
chiuso ognuno\\
nel suo silenzio\\
è teatro in movimento\\
d'una vita malinconica.\\
In ogni particella\\
di luce che violenta\\
le mie oscurità,\\
cerco un ritmo illogico\\
di angolazioni fluorescenti\\
che possano tracciare\\
come curve di Bézier\\
linee guida di un giorno\\
che volge alla morte\\
sperando in una logica\\
necessaria interpretazione\\
sufficiente a confutare\\
la tesi d’una vita inutile.
\end{verse}

\cleardoublepage
\section*{Frammenti}
\vspace{2em}
\begin{verse}
ansanti\\
nell'inutile frenetico errare\\
nella Firenze crepuscolare\\
ci illudiamo l'ultima volta\\
mentre tutto di noi\\
è incastonato nel passato\\
\vspace{1em}
in via Orcagna\\
\vspace{1em}
svogliati\\
nell'illusione del nostro\\
abituale amore distaccato\\
circostanziale cercare\\
la nostra comune risposta\\
ormai priva di desiderio\\
\vspace{1em}
in via Orcagna\\
\vspace{1em}
svuotati\\
nella notte di vento\\
spazzati troppo lontani\\
da tumultuosi sogni\\
condividiamo solamente\\
questo letto a pagamento\\
\vspace{1em}
in via Orcagna\\
\vspace{1em}
inchiodati\\
come la rosa al muro\\
abbiamo visto la genesi\\
del nostro finire\\
e del nostro quadro\\
restano inutili frammenti\\
\end{verse}

\cleardoublepage
\section*{Cade la neve}
\vspace{1.5em}
\begin{verse}
Nell'aria penetrante\\
s'imprime sulla pelle\\
uno spirito gelido\\
che mi sfiora l'anima\\
e sento il gradevole profumo\\
d'un vecchio\\
e sempre nuovo déjà vu.\\
\vspace{1em}
Il tenue cielo fissando,\\
racchiusi nel frammento\\
d'un istante al vento,\\
come primi fiocchi di neve\\
abbracciati,\\
ma ancora prigionieri\\
di indomite correnti,\\
viviamo l'istante\\
prima che l'inverno\\
ammanti ogni sorriso\\
d'un candido oblio.\\
\vspace{1em}
Cade la neve\\
e i nostri sensi\\
ravvivati dal torpore\\
dei legami spezzati\\
vivono la nuova stagione\\
nell'aumentata percezione\\
di noi stessi, insieme.
\end{verse}

\cleardoublepage
\section*{Geometrie}
\vspace{2em}
\begin{verse}
Nel metallico\\
quotidiano crocevia,\\
alzando lo sguardo\\
oltre i limiti\\
del comune pensare,\\
osservo lo stormo,\\
muoversi insieme\\
e spaccarsi\\
e ricomporsi,\\
nelle mie inespresse\\
geometrie mentali,\\
in un cielo di febbraio\\
vasto e freddo,\\
bagnato da un sole\\
che sembra, come me\\
fuori luogo.
\end{verse}

\cleardoublepage
\section*{Nebbia dell'anima}
\vspace{2em}
\begin{verse}
In giorni ritmati dalla malinconia\\
inseguo sagome lontane,\\
smarrito...\\
in una bianca oscurità,\\
vestita d'infinito.\\
\vspace{1em}
Nell'incertezza di ogni passo\\
temendo illogici precipizi\\
cammino...\\
equilibrista a piedi nudi\\
su una cupa lastra tombale\\
scolpita d'angoscia.\\
\vspace{1em}
Dimensione artificiale,\\
percezione del mio personale oblio,\\
plasmata...\\
dal mio respiro\\
nel sibilo del vento,\\
dai miei canti\\
all'ombra dei tuoni,\\
dalle mie lacrime\\
in ogni goccia di pioggia,\\
sei invece immobile valle\\
immersa nella nebbia.\\
\vspace{1em}
E aggrappato al ramo più alto\\
d'un inebriante tiglio in fiore\\
osservo...\\
la mia anima velata\\
vagare solitaria\\
in pensieri ormai senza senso.
\end{verse}

\cleardoublepage
\section*{21 Agosto}
\vspace{2em}
\begin{verse}
Gli Anni dell'effimero viverci\\
stancamente trascorsi\\
nel chiudersi del cerchio\\
di rintocchi equidistanti\\
della tua campana,\\
che anche se non sento,\\
puntuale suona oggi\\
stonata come ieri\\
quando insieme e soli,\\
schiavi dell'altro\\
nelle false certezze del noi\\
incedevamo incatenati\\
ad una logica stringente\\
d'una infelicità costante\\
come archetipo dominante\\
nella tua volta celeste.\\
\vspace{1em}
E Nel tuo giorno\\
d'un agosto bagnato\\
dalla lenta pioggia\\
come autunno in anticipo\\
alla mia porta,\\
foriero di consapevolezza\\
vivendo il ritrovato ordine,\\
ho scavato sereno a mani nude\\
nella calda terra\\
addolcita dal piovasco,\\
e vi ho sepolto\\
l'orrendo clangore,\\
delle opprimenti catene\\
dell'amore in equilibrio\\
sul baratro del compromesso\\
e alimentato unicamente\\
dall'ossimorica fermezza\\
delle tue disinvolte bugie.\\
\vspace{1em}
Ad ogni nuovo rintocco\\
sempre più lontano\\
perso nelle mie profondità,\\
rimane un lontano eco,\\
la tua infelice campana\\
canta come nel giorno dei morti\\
nel cesellato paesaggio\\
creato come cornice\\
ad una ingiallita foto\\
d'un amore trascorso\\
come questa estate\\
che senza il fragore del tuono\\
che senza il vento del cambiamento\\
è comunque già morta.
\end{verse}

\cleardoublepage
\section*{Viaggiare}
\vspace{2em}
\begin{verse}
Immagini rubate\\
di mondi paralleli\\
racchiusi in frammenti\\
di esistenza fermata.\\
\vspace{1em}
E incrociare lo sguardo\\
come a cercare qualcosa\\
dietro alle imposte\\
della vita degli altri.\\
\vspace{1em}
E tornando alla mia\\
mi scopro straniero,\\
tassello anonimo ma unico\\
nel mosaico umanità.\\
\vspace{1em}
E nello scopo d'ognuno,\\
scintilla nella storia,\\
sono cercatore di teoremi\\
tra le nuvole del cielo.\\
\vspace{1em}
E inseguo una logica\\
al ritmo del mio respirare\\
solo quando finalmente\\
ritorno a viaggiare.
\end{verse}

\cleardoublepage
\section*{Dolcemente estranei}
\vspace{2em}
\begin{verse}
Sconvolto dalla recente tempesta,\\
come ad uno scoglio\\
nel mare umanità\\
mi aggrappo indifeso e smarrito,\\
alla desiderata dolcezza\\
in scuri occhi sconosciuti.\\
\vspace{1em}
Donandoci un angolo di volto,\\
solcato nel delinearsi di un sorriso,\\
noi, protagonisti di vite lontane,\\
siamo dolcemente estranei\\
ma uniti in un momentaneo sguardo\\
\end{verse}

\cleardoublepage
\section*{Come l’edera}
\vspace{2em}
\begin{verse}
Vivendo al margine\\
di troppi perché\\
senza cercare\\
alcuna risposta,\\
il tempo si trasforma\\
in un’inesorabile pendolo\\
che consuma\\
il vuoto vivere\\
con una dolce\\
silenziosa oscillazione,\\
e riversa\\
in mari evaporati\\
su colline erose\\
in cieli senza sfumature\\
l’apatia\\
che come l’edera\\
avvolge le rovine\\
di ogni pensiero esacerbato;\\
lasciando che infine la vita\\
sfugga tra le dita\\
come seta d’una bellezza\\
mai voluta scoprire.
\end{verse}

\cleardoublepage
\section*{Padre}
\vspace{2em}
\begin{verse}
Compagno ed inventore\\ 
dei più bei giochi,\\
aiuto e sostegno \\
negli innumerevoli cupi momenti,\\
forte...\\
impulsivo ma non sicuro,\\
per amore eternamente incerto,\\
amico, fratello...\\
\vspace{1em}
Padre\\
Ceppo forte,\\
sostegno sicuro,\\
fonte d’amore\\
dalla quale troppo spesso\\
ho smesso di abbeverarmi\\
pur avendo infinita sete...\\
\vspace{1em}
ma cadi come \\
inesorabile pioggia su di me\\
anche quando da te \\
per orgoglio rifuggo,\\
e dall’assurda lotta\\
fortunatamente sconfitto\\
ritorno a farmi amare.\\
\end{verse}

\cleardoublepage
\section*{Prospettiva}
\vspace{2em}
\begin{verse}
Trasportato nel vento\\
architetto di strutture\\
dalla sfuggevole simmetria\\
tangibile solamente quando\\
le spighe al suo volere,\\
alzando lo sguardo\\
e cambiando prospettiva\\
da ogni coercitivo credo,\\
divengono distese\\
di ritmata perfezione.\\
Inseguendo le correnti\\
d'un inarrestabile cambiare\\
mi sento preda d'un vivere\\
ondeggiante al comune pensiero,\\
cercando l'interiore ampliarsi\\
d'un me stesso difforme\\
dal mare di radici avvizzite\\
troppo ancorate a ideologie\\
d'una stagione senza raccolto.
\end{verse}

\cleardoublepage
\section*{Vite convergenti}
\vspace{2em}
\begin{verse}
Ho bramato occhi scuri\\
come abissi d'un mistero\\
che seducente nello svelarsi\\
potessi sentire un giorno\\
come parte di un nuovo me,\\
vivendo d'uno sguardo,\\
in ogni desiderio riposto\\
nello sfiorare la tua pelle\\
come seta d'un lontano oriente.\\
Attraverso il labirinto\\
dove serenamente smarrito\\
incedo senza fretta\\
con i piccoli passi\\
del quotidiano cercarti.\\
Di bianco vestita,\\
in attesa sui gradini\\
d'una cattedrale di sogni,\\
istantanea d'un incontro\\
innaturalmente cristallizzato\\
dalla percezione d'un "noi"\\
come suggestiva scenografia\\
nel teatro dove siamo attori\\
di due vite convergenti.
\end{verse}

\cleardoublepage
\section*{Tramontare}
\vspace{2em}
\begin{verse}
Nel tardo meriggio,\\
come cipressi piegati\\
dal vento di ponente,\\
perdersi in pensieri\\
d'un sole morente\\
eppur così penetrante,\\
che s'insinua come il dubbio\\
tra le fronde inquiete\\
nell'inarrestabile\\
tramontare.\\
Nell'ultimo raggio,\\
inseguendo le certezze\\
fatte di tenui sfumature\\
d'una vita crepuscolare,\\
come linea tracciata\\
d'un sentiero ritrovato\\
senza volgersi a est\\
per serenamente addentrarsi\\
nella profonda oscurità\\
della propria notte.
\end{verse}

\cleardoublepage
\section*{L’urlo}
\vspace{2em}
\begin{verse}
Il cielo tra le fiamme\\
di un vorticoso tramonto\\
si confonde con il mare\\
tagliato da un ponte infinito\\
\vspace{1em}
è il teatro\\
nel quale l'umanità\\
comunque indifferente\\
distratta cammina\\
sfiorando l’universo\\
spezzato da un vuoto\\
divenuto incolmabile\\
\vspace{1em}
esplode improvviso\\
l’abissale impeto di violenza\\
l’urlo che squarcia il tempo\\
riempiendo ogni spazio\\
\vspace{1em}
e come dolcissima musica\\
il silenzio infine\\
\vspace{1em}
inerte e sgomento\\
il giorno si accascia\\
trascinandosi nell'eco\\
di una misteriosa angoscia
\end{verse}

\cleardoublepage
\section*{Eternamente mia}
\vspace{2em}
\begin{verse}
Esiste un mondo parallelo\\
aldilà della ragione,\\
una vita precedente\\
dove mano nella mano mi conduci\\
mia invisibile compagna,\\
aprendo l'invalicabile cancello\\
di un consapevole subconscio\\
da cui involontarie attingo\\
immagini ricorrenti\\
d'un volto un tempo mio.\\
\vspace{1em}
So che vuoi percepire\\
la mia familiare essenza,\\
anche se schermati\\
dal torpore dei ricordi\\
e da incolmabili spazi,\\
oggi vedo il tuo vivere\\
al ritmo di ogni mio\\
riconquistato respiro,\\
che è comunque per te\\
nel saperti eternamente mia.\\
\vspace{1em}
Ho attraversato\\
quella landa desolata\\
come smeraldina libellula,\\
finalmente liberato\\
dalla paura della morte\\
sono pronto a voltare pagina,\\
reincarnato nella mia terra,\\
scrivendo a chiare lettere\\
sul libro della vita\\
immagini di un nuovo io.
\end{verse}

\cleardoublepage
\section*{Gelide Corsie}
\vspace{2em}
\begin{verse}
Gelide corsie\\
irrealmente immerse\\
in un soffocante\\
prospettico candore\\
artificiale anfiteatro\\
di morente compostezza\\
\vspace{1em}
Porte e poi stanze\\
nell'infinito susseguirsi\\
di sguardi troppo uguali\\
smarriti nella ricerca\\
del senso d'ogni istante\\
vissuto nei ricordi\\
d'una vita ormai passata\\
\vspace{1em}
Asettico purgatorio\\
illuminato dal bagliore\\
intermittente del neon\\
progressiva cognizione\\
delle inutili meccaniche\\
del vivere scivolando\\
istante dopo istante\\
in un abisso\\
fatto di sguardi velati\\
perdendosi nei silenzi\\
della stagione giunta\\
\vspace{1em}
Esplicito e immobile\\
dire addio ai ciclici\\
assillanti ossimori\\
che quotidianamente\\
ci spingono in pensieri\\
distanti dalla vita\\
che inevitabile invece\\
sfugge tra le mani.
\end{verse}

\cleardoublepage
\section*{Parco Massari}
\vspace{2em}
\begin{verse}
Un sole vigoroso\\
inonda di luce\\
il pomeriggio d'una lunga\\
giornata di primavera\\
...\\
le fronde degli alberi\\
ondeggiano seguendo\\
l'armonia del canto \\
degli uccellini\\
...\\
la vita si manifesta\\
in ogni istante\\
e in ogni angolo\\
di verde rinato\\
...\\
un ecosistema di luce\\
e profumata armonia\\
accoglie ogni passo\\
del nostro camminare\\
...\\
mano nella mano
\end{verse}

\cleardoublepage
\section*{Intravedere}
\vspace{2em}
\begin{verse}
Vagando nelle profondità\\
d'un'intera lunga vita\\
per cercare immagini\\
da riportare alla luce\\
per oscurare la solitudine\\
di giorni d'immenso vuoto\\
lasciato come il suo posto\\
da sempre accanto al tuo.\\
\vspace{1em}
per oscurare la solitudine\\
di giorni d'immenso vuoto\\
lasciato come il suo posto\\
da sempre accanto al tuo.\\
\vspace{1em}
E non senti la voce\\
né la mano sfiorarti,\\
segni tangibili del vostro\\
quotidiano accompagnarvi\\
e sostenervi nel vivere.\\
\vspace{1em}
Ma insieme crescendo\\
giorno dopo giorno\\
fortificati dalla vita stessa\\
per giungere con dignità\\
e nel pieno della vostra forza,\\
la saggezza,\\
alla separazione terrena.\\
\vspace{1em}
Raccogli dunque il frutto\\
d'amore vero vissuto\\
riempiendo il vuoto\\
di armoniosa fede\\
che ci rende capaci\\
di intravedere oltre...\\
\vspace{1em}
Ripercorrendo infinite volte\\
la linea del nostro esistere\\
possiamo imprimere\\
una sfumatura di speranza\\
nella vita e nell'amore eterni.
\end{verse}

\cleardoublepage
\section*{Nelle linee dei sogni}
\vspace{2em}
\begin{verse}
Attraverso una fioca luce\\
torbide visioni\\
di parallelismi crepuscolari,\\
come incalzanti sussurri\\
scorgo illusioni di felicità,\\
trattenute oltre\\
queste socchiuse palpebre\\
prigioniere del dormiveglia.\\
\vspace{1em}
La menzognera sera\\
il nero è giunto a coprire\\
come coltre dell'anima,\\
nell'estatico inizio\\
vengo trasportato in parte\\
oltre la soglia\\
nell'immobile forma,\\
che silenziosa si divincola\\
nelle linee dei sogni\\
\vspace{1em}
laconica e tenue alba,\\
nell'illusione già infranta,\\
nel quotidiano rinascere\\
più vecchio d'un giorno\\
e gettato ricorsivamente\\
nel distratto incedere\\
tra lineamenti senza volti\\
alla ricerca d'un sorriso\\
come stella polare\\
del mio serotino naufragare.
\end{verse}

\cleardoublepage
\section*{Rincorrendo}
\vspace{2em}
\begin{verse}
In estate la mente,\\
cercando un sorriso\\
come gemma caduta\\
nel mare dei ricordi,\\
dipinge un quadro naif,\\
teatro di silenziosa\\
lenta caduta di neve,\\
coprendo i tetti\\
ed il mondo intorno\\
tra comignoli fumanti\\
e bambini giocare,\\
goffi ma felici\\
nei loro molti vestiti.\\
\vspace{1em}
E nel pungente inverno\\
accarezzo il tepore,\\
contemplato nell'alto\\
come arabesco tracciato\\
nella volta del cielo,\\
di un sole rovente\\
dominare solenne\\
un bucolico teatro d'afa,\\
canto di cicale\\
tra i tigli in fiore\\
e campi dopo l'aratura,\\
gustando il profumo della terra\\
assalita da rondini affamate.\\
\vspace{1em}
nella volta del cielo,\\
di un sole rovente\\
dominare solenne\\
un bucolico teatro d'afa,\\
canto di cicale\\
tra i tigli in fiore\\
e campi dopo l'aratura,\\
gustando il profumo della terra\\
assalita da rondini affamate.\\
\vspace{1em}
Un fuoco abbandonato\\
senza legna da ardere\\
in una stanza ormai buia...\\
una sedia vuota\\
nella quiete dell'aia\\
nel tardo meriggio...\\
come un fiume di parole\\
frenate dalla diga\\
delle mie paure...\\
contemplo la solitudine,\\
rincorrendo malinconicamente\\
la giostra delle stagioni,\\
e sopravvivendo così al tempo\\
con immagini impresse nell'istinto.
\end{verse}

\cleardoublepage
\section*{Non saper amare}
\vspace{2em}
\begin{verse}
Vittima e artefice\\
di amore intangibile\\
privata della necessità\\
di immergerti in me,\\
navighi quotidianamente\\
in fiumi di parole\\
e mari di comodi silenzi,\\
venendo inevitabilmente\\
prosciugata dal sole\\
d'un egocentrismo zenitale,\\
come spada di Damocle\\
la quale dopotutto\\
nemmeno ti scalfisce.\\
\vspace{1em}
E vagando ciecamente\\
assordata dal silenzio\\
nel soffocante deserto\\
del tuo orgoglio,\\
vittoriosa affondi\\
cullata dal contesto\\
di un mondo di gente sola,\\
dove ognuno si consola\\
nel riflesso di se stesso,\\
specchio come muro\\
creato dall'incapacità\\
di vedere oltre,\\
privandosi dell'altro.\\
\vspace{1em}
Infine, saziata\\
da nuovi propositi\\
proiettando in te stessa,\\
ingannevoli risposte\\
alla tua superficialità,\\
rimani ancora nascosta\\
nel tuo piccolo angolo,\\
sempre più stretto,\\
stretta nella morsa\\
dei tuoi sempre\\
più infiniti limiti,\\
serenamente consapevole\\
di non saper amare.
\end{verse}

\cleardoublepage
\section*{Equilibrio}
\vspace{2em}
\begin{verse}
L'esistenza come sfera\\
nella linea dei giorni\\
con infinite rotazioni\\
come noi stessi\\
mutati e immutati\\
giorno dopo giorno\\
dagli eventi\\
in un'aleatoria armonia\\
inseguita e mai raggiunta\\
dove dal soggettivo essere\\
estrapolare un oggettivo\\
necessario equilibrio. 
\end{verse}

\cleardoublepage
\section*{Il nostro istante}
\vspace{2em}
\begin{verse}
Immersi ma lontani\\
dal mondo circostante,\\
dolcemente soli\\
smarriti nelle vie\\
della città addormentata,\\
siamo artefici d'un ponte\\
nelle nostre distanze\\
abbandonati alla passione\\
d'una notte d'estate.\\
\vspace{1em}
Cullati dai primi rintocchi\\
del mattino seguente,\\
viviamo il nostro istante...\\
\vspace{1em}
Con illogico proiettarci\\
nella breve eternità\\
d'un tenero abbraccio,\\
in punta di piedi\\
ad occhi chiusi\\
tra le mie braccia,\\
come pittori di baci\\
sfiorandoci le labbra\\
dipingiamo di sfumature\\
il nostro quadro d'amore.
\end{verse}

\cleardoublepage
\section*{Veliero solitario}
\vspace{2em}
\begin{verse}
Granello di sabbia\\
disperso al vento\\
tra moltitudini\\
di anonimi erranti,\\
perduto nella rotta\\
come veliero solitario\\
verso il chimerico abisso\\
del mio pragmatismo sognante.\\
\vspace{1em}
In costante equilibrio\\
tra concretezza e utopia\\
come automa a stati infiniti\\
rimanendo ai miei occhi\\
incapaci di disilludersi\\
un insopportabile enigma\\
capace di deludere\\
sopraffatto della vita\\
e ancora raramente...\\
stupire.
\end{verse}

\cleardoublepage
\section*{Ombra al vento}
\vspace{2em}
\begin{verse}
Ho seguito un'ombra\\
in un vicolo cieco\\
unico riparo dalle vie\\
inondate dal sole vivo\\
d'un pomeriggio\\
d'estate inoltrata\\
mentre dell'amore che era\\
non rimane nell'aria\\
che lo stordente profumo\\
della dolce frutta\\
maturata oltre\\
\vspace{1em}
...ora...\\
\vspace{1em}
siamo in quell'ombra\\
perduta la tenerezza\\
perduta la speranza\\
perduti noi stessi\\
\vspace{1em}
Affacciati inevitabilmente\\
al nostro tramonto\\
mi volgo al vento australe\\
parlandogli di un amore\\
bandiera d'un tempo andato\\
nelle inutili dietrologie\\
pregne di agrodolce vergogna\\
verso le necessarie certezze\\
sulle quali costruivo castelli\\
dalle fragili fondamenta\\
ancorate al nostro mentirci\\
\vspace{1em}
...ora...\\
\vspace{1em}
siamo quel vento\\
non si può deviare\\
non si può domare\\
non si può amare
\end{verse}

\cleardoublepage
\section*{Una nuova certezza}
\vspace{2em}
\begin{verse}
Vivendo al margine\\
dell'intersezione\\
del nostro personale\\
concetto d'amore\\
intricato cammino\\
tra i nostri quotidiani\\
mutevoli invarianti,\\
trovo una nuova certezza\\
come mio dilagante\\
sconfinato orizzonte\\
celato nel mistero\\
d'ogni tuo sguardo per me,\\
e in ogni insignificante\\
sconvolgente sfumatura\\
del tua anima,\\
alla mia intrecciata,\\
diventa un gioco\\
di baci ed equilibrio\\
nei tuoi brevi sospiri\\
istanti in cui viverci\\
in questa primavera\\
nella quale siamo\\
come verdi foglie\\
brevi sospiri d’alberi\\
che riassaporano la vita.
\end{verse}

\cleardoublepage
\section*{Autunno sommerso}
\vspace{2em}
\begin{verse}
L'incedere leggero\\
d'ogni stagione\\
come invisibile ombra\\
di antiche visioni\\
ammanta ogni cosa\\
nell'indifferenza\\
dell'animo mio\\
che ancora fugge\\
incatenato ai ricordi\\
di un eterno autunno\\
sommerso dai pensieri\\
di un ocra oceano\\
come foglie avvizzite\\
di adunchi alberi\\
nostre fragili vite\\
spezzate dalla fredda\\
carezza del vento\\
lentamente cadiamo\\
e silenziosamente\\
nel morire nostro\\
doniamo al mondo\\
l'armoniosa danza\\
di nuove stagioni.
\end{verse}

\cleardoublepage
\section*{Oceano d’orgoglio}
\vspace{2em}
\begin{verse}
E scendere nel precipizio\\
di pensieri e acque torbide\\
pregne di incubi lontani\\
che echeggiano\\
come frastuono di tuoni.\\
\vspace{1em}
E aggrapparsi disperatamente\\
ad anelli di una catena\\
di pensieri ancorati\\
al fondo di un oceano\\
di sola angoscia.\\
\vspace{1em}
E affondare per cercare\\
ancora una volta\\
tra i resti del relitto\\
di un amore distrutto\\
dalla tempesta della vita.\\
\vspace{1em}
E affogare ogni notte\\
nel nero turbinio\\
della mente\\
legata al macigno\\
d'un qualcosa d'irrisolto.\\
\vspace{1em}
E come unico rifugio\\
dopo la risalita mattutina\\
circondato\\
da un oceano d’orgoglio\\
c'è un'isola di malinconia.
\end{verse}

\cleardoublepage
\section*{Vita}
\vspace{2em}
\begin{verse}
pianta rigogliosa\\
dalle salde radici\\
sono il ceppo\\
al lento e caldo\\
fuoco della vita\\
che mi consuma\\
in ceneri leggere\\
disperse al vento\\
della dimenticanza
\end{verse}

% ---------- FINE LIBRO ----------
\cleardoublepage

\end{document}
